\documentclass{beamer}

\usepackage[english]{babel}
\usepackage[utf8]{inputenc}
\usepackage{amsfonts}
\usepackage{amssymb}
\usetheme{Copenhagen}
\usepackage{amsthm}
\usepackage{amsmath}
\usepackage{amsopn} 
\usepackage{amsmath,amssymb,lmodern}
\usepackage{pgf}  
\usepackage{textpos}

\addtobeamertemplate{navigation symbols}{}{%
    \usebeamerfont{footline}%
    \usebeamercolor[fg]{footline}%
    \hspace{1em}%
    \insertframenumber/\inserttotalframenumber  
}
%\logo{\pgfputat{\pgfxy(-1,7)}{\pgfbox[center,base]{\includegraphics[width=1.0cm]{logo.png}}}}

\addtobeamertemplate{frametitle}{}{%
\begin{textblock*}{100mm}(\textwidth,-1cm)
\includegraphics[height=1cm,width=1cm,keepaspectratio]{logo.png}
\end{textblock*}}

\title{X-Ray Spectroscopy}
\subtitle{Kvantfysik 2018-2019}
\author[Emma Lorenzo Casas \&  Xabier García Andrade]{\includegraphics[height=3cm,width=3cm]{logo.png}\\Emma Lorenzo Casas \&  Xabier García Andrade}



\begin{document}
	\frame {
		\titlepage
	}
	\frame {
		\frametitle{Theory}
		\begin{columns}[T]
     	\begin{column}[T]{5 cm}
     	\begin{block}{Production of X-rays}
     	X-rays are emitted when outer-shell electrons fill a vacancy in the atom´s inner shell.
     \end{block}
     	\begin{exampleblock}{Characteristic X-rays}
     	Each element releases x-rays in a characteristic pattern.
     	\end{exampleblock}
     	\begin{block}{Need of screening constant}
     	Interaction among electrons
     	\end{block}
     \end{column}
     \begin{column}[T]{7cm} % alternative top-align that's better for graphics
        \begin{alertblock}{Moseley´s law}
        $$ \sqrt{E/R_{y}} = (Z-c) \sqrt{ \left(\frac{1}{n_1^2} - \frac{1}{n_2^2}\right)} $$
        Grants the energy released by each transition.
        \end{alertblock}
        \begin{figure}
        \includegraphics[height=3cm]{energy_dia.png}
        \caption{\tiny Energy level diagram. Image from: http://pd.chem.ucl.ac.uk/pdnn/inst1/xrays.html}
        \end{figure}
     \end{column}
     \end{columns}
	}
	\normalsize
	\frame{
		\frametitle{Methods}
		\begin{columns}[T] % contents are top vertically aligned
     \begin{column}[T]{5cm} % each column can also be its own environment
     \begin{itemize}
     \item Setting up panel control and software
     \item Calibration procedure
     \item Measurement of samples
     \end{itemize}
     \begin{block}{Measuring procedure}
     \begin{itemize}
     \item Turn on the high voltage
     \item Adjust current 
     \item Wait for the measurement to finish and calculate peak center
     \end{itemize}
     \end{block}
     \end{column}
     \begin{column}[T]{7cm} % alternative top-align that's better for graphics
          \includegraphics[height=4cm]{x_ray_method.png}
          
     \end{column}
     \end{columns}
		}

	\frame{
		\frametitle{Results}
		\framesubtitle{Known samples}
		\begin{columns}[T]
		\begin{column}[T]{7 cm}
		\begin{exampleblock}{Silver sample}
		\includegraphics[height= 5cm]{spect_ag.png}
		Tabulated value: $$ K_{\alpha} = 22.17 \hspace{0.2cm} keV \hspace{0.5cm} \epsilon = 0.18 \%$$
		\end{exampleblock}
		\end{column}
		\begin{column}[T]{7 cm}
		\begin{exampleblock}{Zirconium sample}
		\includegraphics[height= 5cm]{spec_zr.png}
		Tabulated value: $ K_{\alpha} = 15.77 \hspace{0.2cm} keV \hspace{0.25cm} \epsilon = 0.95 \% $
		\end{exampleblock}
		\end{column}
		\end{columns}
	}
	\frame{
		\frametitle{Results}
		\framesubtitle{Unknown samples}
		\begin{columns}[T]
		\begin{column}[T]{7 cm}
		\begin{exampleblock}{6th sample}
		\includegraphics[height= 4.5cm]{spect_6th_sample_au.png}
		Tabulated values: 
		\tiny	
		$ K_{\alpha}^{Ge} = 9.88 \hspace{0.25cm} keV \hspace{0.5cm} \epsilon =  0.10 \%$
		$$ K_{\alpha}^{Ni} = 7.44 \hspace{0.25cm} keV  \hspace{0.5cm} \epsilon = 1.47 \%$$
		$$ K_{\alpha}^{Br} = 11.92 \hspace{0.25cm} keV \hspace{0.5cm} \epsilon = 2.18 \% $$
		\normalsize
		\end{exampleblock}
		\end{column}
		\begin{column}[T]{7 cm}
		\begin{exampleblock}{Coins}
		\includegraphics[height= 4.5cm]{spect_coin.png}
		Tabulated value: $K_{\alpha}^{Cu} = 8.04 \hspace{0.25 cm} keV \hspace{0.3cm} \epsilon = 1.99 \%$
		\end{exampleblock}
		\end{column}
		\end{columns}
	}
		\frame{
		\frametitle{Verification of Moseley Law}
		\begin{columns}[T]
		\begin{column}[T]{7cm}
		\includegraphics[height= 6cm]{moseley_law.pdf}
		\begin{block}{Comparison with tabulated value}
		Relative error  $\epsilon = 2.60 \%$
		\end{block}
		\end{column}
		\begin{column}[T]{4cm}
		\tiny
		$$y=a \cdot x + b$$
		$$a=(0.01047 \pm 0.00030) \hspace{0.2cm} keV^{-1}$$ $$ b=(0.27 \pm 0.41) \hspace{0.2cm} keV$$
		\normalsize
		\begin{exampleblock}{Rydberg constant}
		$$a = R_{y} * \left(\frac{1}{n_1^2} -  \frac{1}{n_2^2} \right)$$
		$$u_{(R_{y})} = \frac{u_{(a)}}{\left(\frac{1}{n_1^2}-\frac{1}{n_2^2}\right)}$$
		
		$$R_{y} = (13.958 \pm 0.040) \hspace{0.2cm} eV$$
		\end{exampleblock}
		\end{column}
		\end{columns}

	}
	\frame{
		\frametitle{Discussion of the results and conclusions}
		\begin{columns}[T]
		\begin{column}[T]{5.5 cm}
		\begin{block}{Accuracy of the method}
		Correct prediction of known samples.
		\end{block}
		\begin{exampleblock}{Source of errors}
		\begin{itemize}
		\item Contaminated samples
		\item Incorrect Rate
		\item Noise
		\end{itemize}
		\end{exampleblock}
		\end{column}
		\begin{column}[T]{5.5 cm}
		\begin{alertblock}{Why did we choose these materials?}
		After calculating the relative error, these materials best reproduced the accuracy of the method.
		\end{alertblock}
		\end{column}
		\end{columns}
	
	}

\end{document}